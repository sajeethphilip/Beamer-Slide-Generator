\documentclass[aspectratio=169]{beamer}
\usepackage{hyperref}
\usepackage{graphicx}
\usepackage{amsmath}
\usepackage{tikz}
\usepackage{pgfplots}
\pgfplotsset{compat=1.18}
\usepackage{animate}
\usepackage{multimedia}
% Set the logo to appear on all slides
\logo{\includegraphics[width=1cm]{logo.png}}
\usepackage{url}
\usepackage[export]{adjustbox}
% Add these to your preamble if not already present
\usetikzlibrary{shapes.geometric, positioning, arrows.meta, backgrounds, fit}
% Redefine the frame to have smaller margins
\setbeamersize{text margin left=5pt,text margin right=5pt}
% Centering frame titles
\setbeamertemplate{frametitle}[default][center]
% Set up a dark theme
\usetheme{Madrid}
\usecolortheme{owl}
% Custom colors
\definecolor{myyellow}{RGB}{255,210,0}
\definecolor{myorange}{RGB}{255,130,0}
\definecolor{mygreen}{RGB}{0,200,100}
\definecolor{myblue}{RGB}{0,130,255}
\definecolor{mypink}{RGB}{255,105,180}
% Define new commands for highlighting
\newcommand{\hlbias}[1]{\textcolor{myblue}{\textbf{#1}}}
\newcommand{\hlvariance}[1]{\textcolor{mypink}{\textbf{#1}}}
\newcommand{\hltotal}[1]{\textcolor{myyellow}{\textbf{#1}}}
% Customize beamer colors
\setbeamercolor{normal text}{fg=white}
\setbeamercolor{structure}{fg=myyellow}
\setbeamercolor{alerted text}{fg=myorange}
\setbeamercolor{example text}{fg=mygreen}
\setbeamercolor{background canvas}{bg=black}
\setbeamercolor{frametitle}{fg=white,bg=black}

% Setup short institution name for footline if provided
\makeatletter
\def\insertshortinstitute{airis4D}
\makeatother

% Modify footline template to use short institution
\makeatletter
\setbeamertemplate{footline}{%
  \leavevmode%
  \hbox{%
    \begin{beamercolorbox}[wd=.333333\paperwidth,ht=2.25ex,dp=1ex,center]{author in head/foot}%
      \usebeamerfont{author in head/foot}\insertshortauthor~~(\insertshortinstitute)%
    \end{beamercolorbox}%
    \begin{beamercolorbox}[wd=.333333\paperwidth,ht=2.25ex,dp=1ex,center]{title in head/foot}%
      \usebeamerfont{title in head/foot}\insertshorttitle%
    \end{beamercolorbox}%
    \begin{beamercolorbox}[wd=.333333\paperwidth,ht=2.25ex,dp=1ex,right]{date in head/foot}%
      \usebeamerfont{date in head/foot}\insertshortdate{}\hspace*{2em}%
      \insertframenumber{} / \inserttotalframenumber\hspace*{2ex}%
    \end{beamercolorbox}}%
  \vskip0pt%
}
\makeatother

\title{What Can Be Done with \textcolor{green}{Beamer Slide Generator}?}
\subtitle{Demo}
\author{Ninan Sajeeth Philip}
\institute{\textcolor{mygreen}{Artificial Intelligence Research and Intelligent Systems (airis4D),\\Thelliyoor 689544, Kerala, India.\\\url{http:///airis4d.com}}}
\date{\today}

\begin{document}
\begin{frame}
\titlepage
\end{frame}


\begin{frame}{\Large\textbf{An Introduction To Overleaf}}
    \begin{columns}[T]
        \column{0.5\textwidth}
        \centering
        \fbox{\includegraphics[width=0.95\textwidth,height=0.6\textheight,keepaspectratio]{media_files/How_to_Write_a_Scientific_Journal_Article_Using_Ov_preview.png}}

        \vspace{0.5em}
        \footnotesize{Click to play}
        \movie[externalviewer]{\textcolor{blue}{\underline{Play}}}{./media_files/How_to_Write_a_Scientific_Journal_Article_Using_Ov.mp4}

        \column{0.5\textwidth}
        \begin{itemize}
            \item Overleaf is a complete DTP platform in LaTeX widely used by the academic community
            \item The standalone platform allows users to develop and keep their ideas available online.
            \item It generates the PDF file and can also provide a Rich text-like platform for those who are not very familiar with LaTeX
            \item It is Free.
        \end{itemize}
        \vfill  % Push the rule and source to the bottom of the content
        \hspace{1em}\rule{0.4\textwidth}{0.4pt}\newline\hspace{1em}{\tiny Source: https://www.youtube.com/watch?v=58CoXgze71Y} 
    \end{columns}
\end{frame}


\begin{frame}{\Large\textbf{Google Colab}}
    \begin{columns}[T]
        \column{0.5\textwidth}
        \centering
        \fbox{\includegraphics[width=0.95\textwidth,height=0.6\textheight,keepaspectratio]{media_files/Practical_Introduction_to_Google_Colab_for_Data_Sc_preview.png}}

        \vspace{0.5em}
        \footnotesize{Click to play}
        \movie[externalviewer]{\textcolor{blue}{\underline{Play}}}{./media_files/Practical_Introduction_to_Google_Colab_for_Data_Sc.mp4}

        \column{0.5\textwidth}
        \begin{itemize}
            \item It has everything that is needed to learn and get introduced to AI and Data Science
            \item It is Free
        \end{itemize}
        \vfill  % Push the rule and source to the bottom of the content
        \hspace{1em}\rule{0.4\textwidth}{0.4pt}\newline\hspace{1em}{\tiny Source: https://www.youtube.com/watch?v=oCngVVBSsmA} 
    \end{columns}
\end{frame}


\begin{frame}{\Large\textbf{How Did I Generate This Slide?}}
    \begin{columns}[T]
        \column{0.5\textwidth}
        \centering
        \fbox{\includegraphics[width=0.95\textwidth,height=0.6\textheight,keepaspectratio]{BSG_live.png}}
        \column{0.5\textwidth}
        \begin{itemize}
            \item I just called \textcolor[RGB]{255,165,0}{python BeamerSlideGenerator.py}
            \item Selected option 2 and entered the text file:  demo-presentation.txt
            \item Press enter, and everything is ready. Finally you only need to run \textcolor[RGB]{255,165,0}{pdflatex demo-presentation.tex}
            \item The text file remains unchanged, and I often spell check and grammar check it after typing. Rerun it, and the new presentation will be ready!
        \end{itemize}
    \end{columns}
\end{frame}


\end{document}

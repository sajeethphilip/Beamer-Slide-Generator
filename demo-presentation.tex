\documentclass[aspectratio=169]{beamer}

% Essential packages (core)
\usepackage{hyperref}
\usepackage{graphicx}
\usepackage{amsmath}
\usepackage{tikz}
\usepackage{pgfplots}
\usepackage{xstring}
\usepackage{animate}
\usepackage{multimedia}

% Extended packages with fallbacks
\IfFileExists{tcolorbox.sty}{\usepackage{tcolorbox}}{}
\IfFileExists{fontawesome5.sty}{\usepackage{fontawesome5}}{}
\IfFileExists{pifont.sty}{\usepackage{pifont}}{}
\IfFileExists{soul.sty}{\usepackage{soul}}{}

% Package configurations
\pgfplotsset{compat=1.18}
\usetikzlibrary{shadows.blur, shapes.geometric, positioning, arrows.meta, backgrounds, fit}

% Original text effects
\newcommand{\shadowtext}[2][2pt]{%
    \begin{tikzpicture}[baseline]
        \node[blur shadow={shadow blur steps=5,shadow xshift=0pt,shadow yshift=-#1,
              shadow opacity=0.75}, text=white] {#2};
    \end{tikzpicture}%
}

\newcommand{\glowtext}[2][myblue]{%
    \begin{tikzpicture}[baseline]
        \node[circle, inner sep=1pt,
              blur shadow={shadow blur steps=10,shadow xshift=0pt,
              shadow yshift=0pt,shadow blur radius=5pt,
              shadow opacity=0.5,shadow color=#1},
              text=white] {#2};
    \end{tikzpicture}%
}

% Conditional definitions based on package availability
\IfFileExists{tcolorbox.sty}{
    \newtcolorbox{alertbox}[1][red]{
        colback=#1!5!white,
        colframe=#1!75!black,
        fonttitle=\bfseries,
        boxrule=0.5pt,
        rounded corners,
        shadow={2mm}{-1mm}{0mm}{black!50}
    }

    \newtcolorbox{infobox}[1][blue]{
        enhanced,
        colback=#1!5!white,
        colframe=#1!75!black,
        arc=4mm,
        boxrule=0.5pt,
        fonttitle=\bfseries,
        attach boxed title to top center={yshift=-3mm,yshifttext=-1mm},
        boxed title style={size=small,colback=#1!75!black},
        shadow={2mm}{-1mm}{0mm}{black!50}
    }
}{}

% Base colors (always available)
\definecolor{myyellow}{RGB}{255,210,0}
\definecolor{myorange}{RGB}{255,130,0}
\definecolor{mygreen}{RGB}{0,200,100}
\definecolor{myblue}{RGB}{0,130,255}
\definecolor{mypink}{RGB}{255,105,180}
\definecolor{mypurple}{RGB}{147,112,219}
\definecolor{myteal}{RGB}{0,128,128}

% Glow colors
\definecolor{glowblue}{RGB}{0,150,255}
\definecolor{glowyellow}{RGB}{255,223,0}
\definecolor{glowgreen}{RGB}{0,255,128}
\definecolor{glowpink}{RGB}{255,182,193}

% Basic highlighting commands
\newcommand{\hlbias}[1]{\textcolor{myblue}{\textbf{#1}}}
\newcommand{\hlvariance}[1]{\textcolor{mypink}{\textbf{#1}}}
\newcommand{\hltotal}[1]{\textcolor{myyellow}{\textbf{#1}}}
\newcommand{\hlkey}[1]{\colorbox{myblue!20}{\textcolor{white}{\textbf{#1}}}}
\newcommand{\hlnote}[1]{\colorbox{mygreen!20}{\textcolor{white}{\textbf{#1}}}}

% Basic theme setup
\usetheme{Madrid}
\usecolortheme{owl}

% Color settings
\setbeamercolor{normal text}{fg=white}
\setbeamercolor{structure}{fg=myyellow}
\setbeamercolor{alerted text}{fg=myorange}
\setbeamercolor{example text}{fg=mygreen}
\setbeamercolor{background canvas}{bg=black}
\setbeamercolor{frametitle}{fg=white,bg=black}

% Progress bar setup
\makeatletter
\def\progressbar@progressbar{}
\newcount\progressbar@tmpcounta
\newcount\progressbar@tmpcountb
\newdimen\progressbar@pbht
\newdimen\progressbar@pbwd
\newdimen\progressbar@tmpdim

\progressbar@pbwd=\paperwidth
\progressbar@pbht=1pt

\def\progressbar@progressbar{%
    \begin{tikzpicture}[very thin]
        \shade[top color=myblue!50,bottom color=myblue]
            (0pt, 0pt) rectangle (\insertframenumber\progressbar@pbwd/\inserttotalframenumber, \progressbar@pbht);
    \end{tikzpicture}%
}

    % Modified frame title template with increased height and better spacing
    \setbeamertemplate{frametitle}{
        \nointerlineskip
        \vskip1ex
        \begin{beamercolorbox}[wd=\paperwidth,ht=4ex,dp=2ex]{frametitle}
            \begin{minipage}[t]{\dimexpr\paperwidth-4em}
                \centering
                \vspace{2pt}
                \insertframetitle
                \vspace{2pt}
            \end{minipage}
        \end{beamercolorbox}
        \vskip.5ex
        \progressbar@progressbar
    }
\makeatother

% Institution setup
\makeatletter
\def\insertshortinstitute{airis4D}
\makeatother

% Footline template
\setbeamertemplate{footline}{%
  \leavevmode%
  \hbox{%
    \begin{beamercolorbox}[wd=.333333\paperwidth,ht=2.25ex,dp=1ex,center]{author in head/foot}%
      \usebeamerfont{author in head/foot}\insertshortauthor~~(\insertshortinstitute)%
    \end{beamercolorbox}%
    \begin{beamercolorbox}[wd=.333333\paperwidth,ht=2.25ex,dp=1ex,center]{title in head/foot}%
      \usebeamerfont{title in head/foot}\insertshorttitle%
    \end{beamercolorbox}%
    \begin{beamercolorbox}[wd=.333333\paperwidth,ht=2.25ex,dp=1ex,right]{date in head/foot}%
      \usebeamerfont{date in head/foot}\insertshortdate{}\hspace*{2em}%
      \insertframenumber{} / \inserttotalframenumber\hspace*{2ex}%
    \end{beamercolorbox}}%
  \vskip0pt%
}

% Additional settings
\setbeamersize{text margin left=5pt,text margin right=5pt}
\setbeamertemplate{navigation symbols}{}
\setbeamertemplate{blocks}[rounded][shadow=true]

% Title setup
\title{What Can Be Done with BSG-IDE?}
\subtitle{Enhanced Demo with Special Effects}
\author{Ninan Sajeeth Philip}
\institute{\textcolor{mygreen}{Artificial Intelligence Research and Intelligent Systems (airis4D)}}
\date{\today}

\begin{document}
\maketitle
\begin{frame}
\titlepage
\end{frame}

\begin{frame}{\Large\textbf{Special Text Effects Demo}}
    \vspace{0.5em}
    \begin{itemize}
        \item Regular text vs \shadowtext{Shadow Text Effect}
        \item Using \glowtext{Glowing Text} for emphasis
        \item Gradient text: \gradienttext[myblue][mypurple]{Important Concept}
        \item Spotlight effect: \spotlight{Key Point}
        \item Highlighting: \hlkey{Important} vs \hlnote{Note This}
    \end{itemize}
\end{frame}

\begin{frame}{\Large\textbf{Alert and Info Boxes}}
    \vspace{0.5em}
    \begin{itemize}
        \item \begin{alertbox}[myred]
        \item Critical Information: Pay attention to this!
        \item \end{alertbox}
        \item \begin{infobox}[myblue]
        \item Additional Information
        \item - Useful for side notes
        \item - Can contain multiple points
        \item \end{infobox}
    \end{itemize}
\end{frame}

\begin{frame}[plain]
    \begin{tikzpicture}[remember picture,overlay]
        \node at (current page.center) {%
            \includegraphics[width=\paperwidth,height=\paperheight,keepaspectratio]{bsg-ide.png}%
        };
        \node[text width=0.8\paperwidth,align=center,text=white] at (current page.center) {
            \Large\textbf{Media Layout: Full Frame}\\[1em]
            \begin{itemize}
                \item {\color{white}\glowtext{Full Frame Mode} uses the entire slide}
        \item {\color{white}Text appears over the image}
        \item {\color{white}Great for impactful visuals}
        \item {\color{white}Best with high-resolution images}
            \end{itemize}
        };
    \end{tikzpicture}
\end{frame}

\begin{frame}{Media Layout: Watermark}
    \begin{tikzpicture}[remember picture,overlay]
        \node[opacity=0.15] at (current page.center) {%
            \includegraphics[width=\paperwidth,height=\paperheight,keepaspectratio]{bsg-ide.png}%
        };
    \end{tikzpicture}
    \begin{itemize}
        \item \spotlight{Watermark Mode} shows faded background
        \item Perfect for subtle branding
        \item Maintains text readability
        \item Professional look and feel
    \end{itemize}
\end{frame}

\begin{frame}{\Large\textbf{Media Layout: Overlay}}
    \begin{tikzpicture}[remember picture,overlay]
        \node[opacity=0.3] at (current page.center) {%
            \includegraphics[width=\paperwidth,height=\paperheight,keepaspectratio]{bsg-ide.png}%
        };
        \node[text width=0.8\paperwidth,align=center,text=white] at (current page.center) {
            \begin{itemize}
                \item {\color{white}\gradienttext[myblue][mypurple]{Overlay Mode} balances image and text}
        \item {\color{white}Semi-transparent background}
        \item {\color{white}Ensures content visibility}
        \item {\color{white}Modern presentation style}
            \end{itemize}
        };
    \end{tikzpicture}
\end{frame}

\begin{frame}{\Large\textbf{Media Layout: Picture-in-Picture}}
    \begin{columns}[T]
        \begin{column}{0.7\textwidth}
            \begin{itemize}
                \item Small image in corner
        \item \hlkey{PiP Mode} keeps focus on content
        \item Useful for reference images
        \item Clean and organized layout
            \end{itemize}
        \end{column}
        \begin{column}{0.28\textwidth}
            \vspace{1em}
            \includegraphics[width=\textwidth,keepaspectratio]{bsg-ide.png}
        \end{column}
    \end{columns}
\end{frame}

\begin{frame}{\Large\textbf{Media Layout: Split Screen}}
    \begin{columns}[T]
        \begin{column}{0.48\textwidth}
            \includegraphics[width=\textwidth,keepaspectratio]{bsg-ide.png}
        \end{column}
        \begin{column}{0.48\textwidth}
            \begin{itemize}
                \item Equal space for image and text
        \item \spotlight{Split Mode} perfect for comparisons
        \item Balanced visual presentation
        \item Professional side-by-side layout
            \end{itemize}
        \end{column}
    \end{columns}
\end{frame}

\begin{frame}{\Large\textbf{Media Layout: Top-Bottom}}
    \vspace{-0.5em}
    \begin{center}
        \includegraphics[width=0.8\textwidth,height=0.45\textheight,keepaspectratio]{bsg-ide.png}
    \end{center}
    \vspace{0.5em}
    \begin{itemize}
        \item Image above, content below
        \item \glowtext{Top-Bottom Mode} for sequential flow
        \item Great for step-by-step explanations
        \item Clear visual hierarchy
    \end{itemize}
\end{frame}

\begin{frame}{\Large\textbf{Media Layout: Corner}}
    \begin{itemize}
        \item Small image in corner
        \item \hlnote{Corner Mode} minimizes distraction
        \item Maintains focus on content
        \item Subtle visual support
    \end{itemize}
    \begin{tikzpicture}[remember picture,overlay]
        \node[anchor=south east] at (current page.south east) {%
            \includegraphics[width=0.2\textwidth,keepaspectratio]{bsg-ide.png}%
        };
    \end{tikzpicture}
\end{frame}

\begin{frame}{\Large\textbf{Multiple Images with Mosaic}}
        \begin{center}
        \begin{tikzpicture}
           \matrix [column sep=0.2cm, row sep=0.2cm] {
               \node { \includegraphics[width=0.4\textwidth,height=0.35\textheight,keepaspectratio]{bsg-ide.png} }; &
               \node { \includegraphics[width=0.4\textwidth,height=0.35\textheight,keepaspectratio]{bsg-ide.png} }; &
               \node { \includegraphics[width=0.4\textwidth,height=0.35\textheight,keepaspectratio]{bsg-ide.png} }; \\
               \node { \includegraphics[width=0.4\textwidth,height=0.35\textheight,keepaspectratio]{bsg-ide.png} }; &\\\\
        };
    \end{tikzpicture}
    \end{center}
    \vspace{0.5em}
    \begin{itemize}
        \item \gradienttext[myblue][mypurple]{Mosaic Layout} combines multiple images
        \item Perfect for comparisons
        \item Grid-based arrangement
        \item Visual impact with organization
    \end{itemize}
\end{frame}

\begin{frame}{\Large\textbf{Interactive Media Demo}}
    \begin{columns}[T]
        \begin{column}{0.48\textwidth}
            \includegraphics[width=\textwidth,height=0.6\textheight,keepaspectratio]{How_to_Write_a_Scientific_Journal_Article_Using_Ov_preview.png}
            \begin{center}
                \vspace{0.3em}
                \footnotesize{Click to play}\\
                \movie[externalviewer]{\textcolor{blue}{\underline{Play}}}{media_files/How_to_Write_a_Scientific_Journal_Article_Using_Ov.mp4}
            \end{center}
        \end{column}
        \begin{column}{0.48\textwidth}
            \begin{itemize}
                \item \spotlight{Video Integration}
        \item Automatic video thumbnail generation
        \item Click to play functionality
        \item \hlkey{Source Attribution} automatic citation\footnote{YouTube video: \href{https://www.youtube.com/watch?v=58CoXgze71Y}{\textcolor{blue}{[Watch Video]}} }
            \end{itemize}
        \end{column}
    \end{columns}
\end{frame}

\begin{frame}{\Large\textbf{Combined Effects Example}}
    \begin{columns}[T]
        \begin{column}{0.48\textwidth}
            \includegraphics[width=\textwidth,keepaspectratio]{bsg-ide.png}
        \end{column}
        \begin{column}{0.48\textwidth}
            \begin{itemize}
                \item \begin{alertbox}[myorange]
        \item \glowtext{Important Visual Concept}
        \item \end{alertbox}
        \item \gradienttext[myblue][mypurple]{Main Point} with gradient effect
        \item \spotlight{Key Concept} with spotlight
        \item Regular bullet point with \hlkey{highlighted term}
        \item \begin{infobox}[myteal]
        \item \shadowtext{Additional Notes:}
        \item - Supporting details
        \item - Extra information
        \item \end{infobox}
            \end{itemize}
        \end{column}
    \end{columns}
\end{frame}

\begin{frame}{\Large\textbf{Code Highlighting Demo}}
    \vspace{0.5em}
    \begin{itemize}
        \item \begin{minted}{python}
        \item def hello\_world():
        \item print("\glowtext{Hello, World!}")
        \item return True
        \item \end{minted}
        \item Code highlighting with \hlkey{minted} package
        \item Supports multiple languages
        \item Syntax highlighting included
    \end{itemize}
\end{frame}

\begin{frame}{\Large\textbf{BSG-IDE Features Summary}}
    \begin{tikzpicture}[remember picture,overlay]
        \node[opacity=0.3] at (current page.center) {%
            \includegraphics[width=\paperwidth,height=\paperheight,keepaspectratio]{bsg-ide.png}%
        };
        \node[text width=0.8\paperwidth,align=center,text=white] at (current page.center) {
            \begin{itemize}
                \item {\color{white}\begin{alertbox}[mygreen]}
        \item {\color{white}\glowtext{Key Features}}
        \item {\color{white}\end{alertbox}}
        \item {\color{white}\gradienttext[myblue][mypurple]{Multiple Media Layouts}}
        \item {\color{white}\spotlight{Special Text Effects}}
        \item {\color{white}\hlkey{Interactive Elements}}
        \item {\color{white}\shadowtext{Professional Formatting}}
        \item {\color{white}\glowtext{Automatic Enhancements}}
            \end{itemize}
        };
    \end{tikzpicture}
\end{frame}

\end{document}
\end{document}
